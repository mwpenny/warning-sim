\chapter{Conclusion}
\label{Conclusion}

In this study, we experimented with the use of tiered warning messages to convey risk to users while browsing the web. We found that our high severity tiered warnings had significantly higher adherence rates and user severity ratings compared to the warnings currently present in Firefox. Conversely, our low and medium severity tiered warnings had significantly lower adherence rates and user severity ratings. On average, participants reacted significantly more quickly to our tiered messages than the control messages (whether that decision was to adhere to a warning or ignore it).

Based on eye tracker data and participant responses, we attribute our results primarily to the use of colour and imagery in our warning messages. Participants were able to easily learn which messages were more severe using the provided cues and quickly make decisions when presented with new warnings of the same severity level. However, the learning was likely facilitated by the presence of the different tiers of warnings therefore it is questionable whether or not the high severity warnings would be as effective if they were presented on their own. Future research is therefore needed.

We suggest that warning message designers include multiple features to convey the risk of a given situation. For example, a dyslexic participant who had trouble reading the messages was able to quickly use colour and imagery instead. This redundancy aids in reinforcing messages and also catering to different users, making warnings more usable and accessable.
