\chapter{Introduction}
\label{Introduction}

Browsing the web entails many risks, and users are frequently faced with dangerous situations. It is important to make them aware of these dangers and reduce the likelihood of negative outcomes. In modern web browsers this is implemented in the form of warning messages which are presented to users in the event of suspected dangerous situations. However, research suggests that the existing messages and methods employed by browsers are not very effective and that users often ignore the warnings \cite{akhawe2013alice, almuhimedi2014reputation, anderson2015polymorphic, bravo2011bridging, dhamija2006phishing, egelman2008warned, sunshine2009crying}.

We believe that there is room for improvement in this area through more varied and easily understandable warning messages. In this study we experiment with the effectiveness of tiered warnings: messages which have varying severity levels and appearances depending on the situation in which they are triggered. We investigate how these new messages impact warning adherence and reaction time and which aspects of them influence user behaviour.

Our goal is to increase warning adherence by conveying the risk associated with dangerous situations more effectively. Conveying such information is difficult, particularly to users without a security background who simply want to browse the web. Common usability practices and past research in this area has shaped our warning message design in an attempt to learn from past mistakes, improve upon what works well, and incorporate new ideas.
